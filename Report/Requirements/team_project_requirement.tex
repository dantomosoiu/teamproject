

\section{System Requirements}

Due to the scope of the project it would be impossible to commence
without a clear vision of how the end product should function. This
also aids greatly in breaking down the system into seperate components
allowing team members to work independently without repetition of
work. These requirements were decided on during the early stages of
the project. The reasoning behind them comes from a number of sources,
primarily -
\begin{itemize}
\item Research into previous attempts at evacuation simulation. The teams
behind these will have each conducted their own research. Rather than
repeat their efforts examination of their choices and rationalisation
provides a valuable and dense insight into what shall be required.
\item Interviews with a target user (The fire warden at our initial site).
Since target users are the potential customers for our product it
would be naive not to consider what they would expect.
\item Requirements derived from research into evacuation and human behaviour.
\item Discussions amongst the team.
\end{itemize}
The Requirements have been split into several sections depending on
their necessity, usefulness and difficulty in implementation.


\subsection{Must Have}

These are the bare minimum requirements for the system to be suitable
for any kind of customer use. Without these the software would either
be considered entirely non-working or faulty. Not managing to provide
these would indicate complete failiure of the project.
\begin{itemize}
\item Representation of a 3D environment in a manner allowing the system
knowledge of navigatable surfaces.
\item Generation and representation of a population within the 3D environment.
\item Ability to manipulate the population to allow movement towards a given
location.
\item Output to the user of time taken for the entire population to successfully
move to a safe location.
\item Basic behavioural and routing algorithms to generate paths of movement
for the population.
\end{itemize}

\subsection{Should Have}

High priority requirements which should be satisfied where possible.
Missing any of these would indicate the project as incomplete and
failing to meet all of its objectives.
\begin{itemize}
\item Accurate behavioural models allowing realistic actions from individuals
within the population.
\item Interaction of members of the population to allow realistic crowd
behaviour and collision control.
\item Ability for the user to specify environmental and population variables
(such as population size).
\item Pseudo-random population generation including individual characteristics
and behavioural patterns as well as initial position.
\item Accurate Timescale for evacuation process
\item Intuitive GUI allowing viewing of both environmental variables and
a visual representation of the environment throughout the evacuation
process.
\item Ability for user to change perspective (camera location) within the
visual representation.
\item Ability to import alternative environmental models.
\item Extensive Documentation and user help.
\end{itemize}

\subsection{Could Have}

Requirements considered desirable but not necessary. To be included
if time and resources allow.
\begin{itemize}
\item Ability to pause and resume the simulation
\item Representation of assurances and hazards within the environment such
as exit signs and low ceilings.
\item Variable run speed of simulation
\item Ability to output information such as evauation time and path of each
member of the population to file allowing further future analysis.
\item Accurate modelling of advanced crowd interation, such as staff assisting
the guest population.
\item Unusual population traits such as disabled guests and the interaction
and assistance required for their safe exit.
\end{itemize}

\subsection{Would Like to Have}

Features unlikely to be present in the initial release due to time
and resource constraints.
\begin{itemize}
\item Accurate fire and smoke model.
\item User control over fire start location.
\item Environmental model support of different materials to allow further
accuracy to the fire model.
\item Advanced and highly realistic graphical representation of environment.\end{itemize}
