\documentclass[a4paper,10pt]{article}
\usepackage[utf8]{inputenc}

%opening
\title{Requirements}
\author{Team L}

\begin{document}

\maketitle

\section{Requirements from previous project (Purges)}
  - Model a population in the environment with basic behaviour.\\
  - Ability to represent hazards/block paths in simulator.\\
  - Ability to load multiple environment models.\\
  - Variable run-speed of simulation.\\
  - GUI for user interaction.\\
  - User ability to modify perspectives.\\
  - Play/Pause/Stop/Restart functions.\\
  - Provide user ability to specify environmental variables.\\
  
\section{Requirement from research}
  - Accurate behavioural model.
  
\section{Requirements of our own}
 - 3D representation of models.\\
 - Accurate timescale for simulation.\\
 - Random population generation.\\
 - Dynamic environment modelling (automatic grid/graph generation).\\
 - Statistical feedback in display.\\
 - Accurate fire model.\\
 - Possibly allow statistical to be exported to a standard file format for analysis.\\
 - Provide user control over fire source (area and cause).\\
 - Represent a logical model of the environment.\\
 
 \section{Requirements from firewarden}
 - Accurately model staff and staff/guest interaction.\\
 - Include wheelchair using guests and interaction required to carry them out.\\
 - Factor in guest reaction time and effect of alarm.\\
 \subsection{How we interview firewarden}
  1. Gather questions to ask firewarden.\\
  2. Record the interview.\\
  3. Ask for suggestions.\\
  

\end{document}
