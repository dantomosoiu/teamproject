
\section{Selection of 3D Environment}

The key decision before implementation could commence was the environment
to work in. This was heavily interlinked with the choice of main
programming language. Due to time constraints, building an entire 3D
engine would be outside the scope of the project.


\subsection{Programming Language Considerations}


\subsubsection{C++}

Designed to be a superset of the C language, supporting the object-orientated
paradigm. It is industry standard for almost all 3D graphics development
\cite{Wilson2006}, bridging the gap between lower level languages
such as C and object orientated languages such as Java and C\#. Some of the advantages of the
language include its speed and power combined with the available 3D libraries and very
high portability. The majority of available engines
are also written in C++ offering a large selection.

However, as a team we had no experience at all with the C++ programming
and at the time of selection only a minor knowledge of C. This alone
would be a steep learning curve, but to provide the 3D environment
required an understanding of either the OpenGL library or Direct3D
(see Graphics Standards below) would have possibly also been required.
There are also those who argue instead of trying to bridge gaps, low
level components of games should be written in C. They argue that due to
the information hiding afforded by C++ it is often easier to write
efficient code in its lower level counterpart\cite{Wilson2006}. Memory
management is also far less advanced than alternatives offering an
easy leeway to memory leaks.

\subsubsection{Java}

Written to have as few implementation dependencies as possible, Java
is incredibly portable\cite{AboutJava}. Rather than to binary, Java
compiles to what is known as byte-code which runs on the Java Virtual
Machine, unrelated to the architecture underneath. The language as
with C++ implements the object-orientated paradigm allowing abstraction
and with standard extensions such as Java3D offers a less intimidating
entrance into 3D.

With powerful free IDEs such as Eclipse and Netbeans
writing in Java becomes quick and painless. Threading is built in
and whilst large, the instruction set is easy to learn. Furthermore, 
Java is the language we as a team have the most experience with. This
would allow implementation of the basic components to begin immediately.
There is also the Java garbage collector. This automatically retrieves
memory which is no longer reachable taking away risks of memory leakage.

The downside to Java is that the abstraction from architecture comes at
a cost of speed. Since there is both virtualisation combined with
a high level environment it is not as efficient when compared to other languages such as C\cite{Jelovic}. 


\subsubsection{Python / A high level scripting language}

The majority of performance issues relating to 3D programming come
from the underlying engine. Since we were anticipating usage of a
pre-existing system, we could then build our system over the top of
this using a high level language such as Python. Using libraries such
as Boost.Python\cite{boostPython} the two languages can be bound,
allowing claimed seamless interoperability. Performance bottlenecks
due to Python could be overcome using C++. Less critical tasks could
be written quickly and easily in Python.

Some of the team had some Python experience. This fact combined with the less intense learning
path than pure C++ meant that this methodology was a strong consideration.


\subsection{Graphics standards}

Whilst this mostly came down to our choice of engine, it was a consideration
to take during our decisions. There are two viable options, Direct3D
and OpenGL.


\subsubsection{Direct3D}

Direct3D is a proprietary API (Application Programming Interface) designed
by Microsoft Corporation. It was created to allow games creators more
open access towards hardware giving much better performance. Whilst
in previous years it suffered performance issues and multiple bugs
it has improved drastically and is now considered by many to be the
industry standard for Windows platforms\cite{Roy2002}.

Its power is also one of its key weaknesses. There is a steeper learning
curve than OpenGL and it takes considerable work just to initialize.
The other big weakness is portability. Support outside of Windows is extremely poor. 
Whilst Wine, a compatibility layer for Unix-based
systems offer mostly functional ports, these are impeded due to dependencies
on other Windows libraries.


\subsubsection{OpenGL}

OpenGL is an open standard API which was for a number of years little
disputed as the industry standard. It is available on a large variety
of platforms including Windows, Mac, and Linux based systems. It provides
a strong range of functionality and was designed to be as future-proof
as possible. There is a proven history of stability and to add to
its core functionality there is the ability for extensions. Since
its future is controlled by a board made up from a large diverse group
of companies its strengths apply to a large number of applications.

Downsides are also many but revolve around two main issues. OpenGL
was built 10 years ago and the future it is was built to work for has
arguably come and gone. Extensions go a way to remedy this but are
hindered by many being vendor specific.


\subsection{3D Engines}

The majority of the 3D functionality we needed could be provided by
an existing engine, either developed for simulation or game purposes.
Concepts required are needed by a wide range of industries making
current developments extensive and abundant.

Due to time constraints it would be difficult to create
a bespoke system able to provide as detailed and efficient performance.
As a result of this we decided to use one of these pre-existing solutions.
This choice would be heavily interlinked with our preferences towards
other tools.

Many of these are games engines. Games engines often are
designed to simulate a real-world environment which offers exactly
what is needed for this project. Due to the many options available,
only a subset are mentioned below.


\subsubsection{Unity}

Unity is a cross-platform engine written in C/C++, however it also
supports code written in C\# and JavaScript. It has its own rendering
capable of using Direct3D or OpenGL. There is strong support for 3D
model importation from a large range of formats. It has its own scripting
language as well as supporting C\# and Boo (which has a syntax inspired
from Python). The basic license would provide all the features we
needed and is free.

The emphasis is on providing incredibly powerful
GUI design tools not games. The logic is aimed to be done
entirely in scripting languages which would give performance issues
when combined with the behavioural processing that would be required.

Whilst C\# can be used, we have no experience with the language. C\#, as with
Java, offers performance shortfalls when compared to C/C++.


\subsubsection{Panda3D}

Panda3D is an open source framework for 3D rendering and development
of programs written in Python and C++. It offers a reasonably powerful
environment with a relatively shallow learning curve. There is a strong
and active community support system but the documentation appears
to be lacking compared to other alternatives.

It is cross platform among the three key operating systems.
It supports both OpenGL and Direct3D providing a relatively thin wrapper
around the lower level APIs.

If the decision was made to take the Python and C++ route, this would be a strong
option to consider.


\subsubsection{jMonkeyEngine}

jMonkeyEngine is designed partially as a games engine and partially as a
replacement for the now unsupported Java3D. Written purely in Java
all the advantages mentioned towards the language above would also
apply here. All recent versions of OpenGL are also fully supported
offering advanced graphics capabilities.

Fundamentally, the project is solely a collection of libraries making
it a low-level tool which would give us the flexibility we would need
considering the majority of our code would be related to the simulation
as opposed to the graphics rendering.

If the decision was made to work in Java, then there is no comparable competition.


\subsubsection{CryENGINE 3}

CryEngine is an advanced engine created by Crytek originally as a
technology demo for Nvidia but the company soon saw its potential.
This has led to massive success with a burst of successful high profile
games based on the engine.

Programming for CryEngine is accomplished using C++. This gives an extremely
powerful combination allowing incredible graphics with a high performance
back-end.

The big downside is the lack of support for OpenGL. As a result there
is little portability outside of Windows. It is also only free for
non-commercial use, meaning if the project were to be taken beyond
the initial research aims an incredibly expensive license would be
required.


\subsubsection{Game Blender}

Blender is a free and comprehensive 3D production suite, one component
of which is a games engine. Considering Blender was a strong contender
for use in our modelling (see Modelling below) there would be no importation
issues. The engine is a mostly independent component written in C++
including support for Python scripting. Whilst a relatively young
project, it offers all the 3D functionality that would be required,
however is lacking in the back-end code support which would be needed.


\subsection{3D Modelling}

To manipulate a 3D environment, such an environment must first be created. This
involves modelling the chosen structure in a way that could be imported
into the physics engine. Since many of the engines considered included
modellers, the decision of what 3D modelling tool to use is linked to the decision of what graphics environment is chosen. 

\subsubsection{Blender}

As mentioned above, Blender is a comprehensive 3D production suite.
Its main usage is in creation of 3D models. A large number of the
games engines we were considering either supported Blenders native
.blend format, or one of the many alternative formats the suite could
export as. There is extensive documentation combined with a strong
and active support community.

The feature-set offered is comparable with some of the widespread industry
tools. Released under a GNU General Public License the software is
free to use.


\subsubsection{Autodesk 3DS Max}

3DS Max is an incredibly powerful suite and the one most used in the
modelling industry. It is comprehensive and versatile but the features come
at a cost. Licenses are extremely expensive and the system requirements
are far from low. Whilst the license cost can be avoided since free
versions are available for students, we would need to gain access
to hardware capable of handling the software.

\subsubsection{Hexagon}

Hexagon, unlike the other options discussed is purely a modelling program.
Offering equal functionality in this area, advanced and detailed models
can be built. The interface is intuitive and easy to learn. The software
also has a very low retail price.

\subsection{Choices}

Whilst the Python and C++ routes were a strong consideration it was
decided the benefits failed to overcome the time required to learn
a completely new language. Although the documentation was not perfect,
jMonkeyEngine appeared to offer all the features we required. It was decided
that, provided the back-end code was written in a reasonably efficient
manner, performance should not be an issue.

Since jMonkeyEngine offered an inbuilt importer, Blender was a complimentary
modelling choice. The minor benefits offered by the proprietary solutions
were far from counterbalancing the restrictions licenses would incur.

