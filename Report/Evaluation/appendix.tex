
\textbf{Tasks}
\begin{enumerate}
\item Set the population size to 50. Populate the ship.
\item Hide the ship frame from view, then show it again.
\item Generate the exit routes for the population
\item Change the camera angle so you have a bird's eye view of the ship.
\item Evacuate the ship. While the evacuation is taking place, change the camera view to face the exits of the ship.
\item Read out loud the time the simulation took and the number of people evacuated.
\end{enumerate}

\textbf{Participant 1 (ID: 146)}
\begin{enumerate}
\item Participant successfully changed the population size to 50 but was unclear about the status of the system after clicking the populate button. Participant questioned whether anything was happening.
\item Did not understand the task fully, particularly what was meant by ‘ship frame’. The participant tried to hide the ship from view by changing the location of the camera so that the ship was not in view but this was not what was wanted.
\item Participant was eventually able to accomplish the task although at one point was confused and thought camera location had an effect of route generation. This did not affect the outcome of the task, though.
\item Controls used correctly but the camera angle was slightly short of the bird's eye angle asked for.
\item Participant knew to use the evacuate button. When changing the camera view, the participant made use of the camera location controls and manually panned the camera to find the exits instead of using the preset camera location drop-down menu. Judging the completion of the simulation was done by noticing the absence of people from the ship and not by the ‘remaining people’ display panel.
\item Completed successfully.
\end{enumerate}

\textbf{Participant 2 (ID: 102)}
\begin{enumerate}
\item Accomplished successfully, although expressed views of uncertainty of system status.
\item No problems completing task.
\item Successfully generated routes although was unsure again about system status. The participant said that they did not know whether the button had worked or not.
\item Participant went straight to the camera location drop-down but realised another method was needed because `bird's eye' was not there. Tried to use the manual controls and the mouse on the simulation environment but couldn’t finish the task.
\item Clicked the correct button for evacuate but when selecting the ‘exits’ item from the drop-down menu, did not immediately click the ‘move’ button to register the action. The participant realised the mistake and clicked the move button to finish the task.
\item Completed successfully.
\end{enumerate}

\textbf{Participant 3 (ID: 93)}
\begin{enumerate}
\item Accomplished without hesitation.
\item Looked at the interface for longer to identify the correct option. At first the participant clicked the drop-down camera locations but realised this was a mistake. The participant was confused as to what the task was asking and assumed ‘navmesh’ was the ship frame.
\item Moved mouse from the top to the bottom of the display to identify the button to press. The participant correctly identified and pressed the button although was unsure about the progress of the system. The participant assumed something was happening in the background because a lot of text was being displayed in the console.
\item The camera locations drop-down was the first thing the participant looked at. The participant could not find the correct option so looked for an alternative. The ‘cam speed’ slider was tried a few times. The participant expected it to do something but nothing was happening on screen. The participant then pressed buttons haphazardly and admitted that they could not finish the task.
\item Successfully evacuated the ship, moved the camera to the exits using the drop-down menu and identified when the evacuation ended.
\item Completed successfully.
\end{enumerate}

\textbf{Participant 4 (ID: 24)}
\begin{enumerate}
\item Population size changed successfully but there was a slight hesitation finding the populate button.
\item Completed successfully although the participant had some confusion over whether the frame asked for in the question and the navmesh were the same thing.
\item Clicked the ‘route’ button but was unsure whether the routes were actually being calculated.
\item Could not find `bird's eye' view in the drop-down menu and was confused over the semantics of the camera control buttons. The participant failed to complete the task.
\item The participant completed the task successfully. The completion of the evacuation was immediately noticed.
\item Completed successfully.
\end{enumerate}

\textbf{Participant 5 (ID: 46)}
\begin{enumerate}
\item Participant correctly changed the population size but thought that on entering the population, the system would populate itself. The participant clicked the populate button but was unclear when the system was finished completing the task.
\item At first the participant did not bring the ship frame back into view but was able to hide it with no problems.
\item Correctly identified the button to press but was unclear about the system state.
\item Participant was able to use the mouse and the buttons on screen to manipulate the camera angle. It was noted that the participant went to the on screen buttons first and then used the mouse on the simulation canvas to finish the task.
\item The participant panned to the exits using the manual buttons and not the preset camera location drop-down menu as anticipated. The participant also announced the end of the evacuation prematurely by not looking at the ‘remaining people’ metric at the right hand side.
\item Completed successfully.
\end{enumerate}

\textbf{Participant 6 (ID: 81)}
\begin{enumerate}
\item Completed successfully but was unsure of system state.
\item The participant had a slight hesitation to find the correct button to press but accomplished the task nevertheless.
\item The route button was found clearly by the participant although it was remarked that nothing was happening on screen when there was really something happening in the background.
\item The participant went straight to control the ship with the mouse directly interacting with the canvas. This was supplemented by use of the camera control buttons provided on the interface and the task was completed successfully.
\item The participant did not notice the drop-down menu item to set the view to the exits view already defined. Instead, the manual controls were used to pan to the exit location. The ‘remaining participants’ metric in the sidebar was not used to determine whether the evacuation was finished. Instead, the participant preferred to see whether there were any people still on the ship by looking at the simulation graphics.
\item Completed successfully.
\end{enumerate}

\textbf{Participant 7 (ID: 334)}
\begin{enumerate}
\item Participant thought it was quite obvious how to accomplish the task and completed it without any trouble. The system status was not clear to the participant.
\item The terminology of ‘ship frame’ versus ‘navmesh’ was questioned and assumed by the participant to mean the same thing. The task was completed after this initial hesitation.
\item Participant remarked that the button was called ‘route’ and though that it should be called ‘routes’ to indicate more than one route. The participant was not sure whether anything was happening and was confused about the overlapping lines which appeared on the screen after this task was accomplished.
\item Participant went straight to the camera locations drop-down to find a preset. After noticing that there was no preset, the participant proceeded to instinctively use the mouse and a combination of the on screen buttons to accomplish the task.
\item Successfully identified and clicked the correct button and then proceeded to investigate the scene using the mouse and the scroll on the mouse. The participant raised some questions such as how many exits there are and how the user is supposed to know where a particular exit is.
\item Completed successfully.
\end{enumerate}

\textbf{Participant 8 (ID: 336)}
\begin{enumerate}
\item The participant highlighted the text box and changed the population size without any difficulty.
\item The terminology of “frame” confused the candidate and wasn’t sure whether she had done the correct thing by hiding and showing again the navmesh.
\item The participant was not sure whether the route button was the correct button to press. It was assumed by the participant to be correct after noticing that more things appeared on the screen.
\item The camera location drop-down was selected first and the participant selected exits to see what it did. This was the wrong thing to do and realising the error, the participant proceeded to press all the camera location buttons to try to complete the task. There was not much structure to the way the buttons were pressed. On discovery of the fact that the mouse could be used on the simulation canvas, the task was completed successfully.
\item The participant was not sure they had to click the ‘move’ button to register the camera move event but eventually realised this had to be done.
\item Completed successfully.
\end{enumerate}
