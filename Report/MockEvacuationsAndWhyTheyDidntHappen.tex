\Section{Evaluating the Behavioural Model}
When attempting to evaluate the effectiveness of a behavioural model implementation, in general all approaches involve a real-world enactment in the form of a mock evaluation. Alternatively, it is possible to replicate a previous incident where a real evacuation occured and try to recreate this in the system. Both of these approaches have severe drawbacks.\\

At first glance a mock evaluation may seem a tangible approach. However it has the following major drawbacks:
\begin{enumerate}
\item{\textbf{Cost:} one of the key reasons for building an evacuation simulator is to avoid the need for mock evacuations, as the cost of shutting down the building and/or hiring mock evacuees can be hindering. Obviously this creates a paradox: the system is built to remove the need for mock evacuations but mock evacuations are needed to evaluate it.}
\item{\textbf{Lack of Urgency:} many evacuation behaviours are driven by the individual's urgency to exit. In a mock evacuation, there is obviously no real urgency for participants to exit as there would be in a real evacuation. This is difficult to account for in results, since the extent to which urgency changes behaviour may vary so greatly between individuals.For this reason it is unlikely to observe any competitive or otherwise aggressive behaviour, again since there is no sense of urgency to drive such actions.}
\item{\textbf{Danger:} even if we assume that we can replicate a sense of urgency, this may expose participants to unacceptable danger. For example we may expect in a real evacuation that an individual may be trampled, but naturally this is not something we wish to have happen in an experiment. The risk of such incidents occuring outweighs the benefits.}
\end{enumerate}
It had been hoped that a mock evacuation carried out by the staff of the Tall Ship could be observed for some comparison data, but this could not be arranged. In conclusion no evaluation of the behavioural model could be carried out, and it is therefore impossible to say to what extent it is an accurate model.

